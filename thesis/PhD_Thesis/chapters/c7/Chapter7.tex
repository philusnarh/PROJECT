\chapter{Conclusions and Recommendations} % Main chapter title

\label{Chapter7} % For referencing the chapter elsewhere, use \ref{Chapter1} 

IM is very sensitive to the aggregate line emission of galaxies other than the limiting magnitude of traditional galaxy surveys,
which focus on only discrete sources whose emission lies above some flux threshold, defined within a limited aperture. 
As such it represents a promising technique for statistical studies of galaxies that are faint. However, to extract the meaningful information we have
to address the exact subtraction of the foreground continuum signal from Galactic and extra-galactic radio sources.
% the confusion problems caused by interloping lines from foreground galaxies.
The main focus of this work is to develop IM techniques for mapping out primary beams of a radio telescope and then, introduce realistic errors to perturb these modelled beams.
We then attempt a correction and calibration of these distorted modelled beams and ultimately, use the final data for IM experiments. Thus, we use
these modelled beams to simulate  the full-sky polarization maps and then, determine the amount of foregrounds that have corrupted the total intensity due to 
polarization leakage and errors in the primary beams which have not been accounted for. Therefore, the two critical things the study addresses are: 
\begin{enumerate}[label=(\roman*)]
 \item the contribution of polarization leakage to the measured HI and CO power spectrum, given some more or less realistic primary beams; and \label{itm:first}
 \item the uncertainty on the estimate of~\ref{itm:first} introduced by unmodelled perturbations in the primary beam.
\end{enumerate}

\noindent The following are some of the main findings of the research:
\begin{itemize}
% \item  The IXR computed for the SKA1-mid shows that the angular separation for which the magnitude of the beam pattern decreases is approximately $-20$ dB 
% from the peak of the main beam.
\item A realistic primary beam pattern can be simulated with the {\tt OSKAR} software by modelling a single dish as a collection of dipoles, such that the orientation 
of the dipoles are uniformly distributed across the radius of the dish.
\item Zernike fitting is a good model to reconstruct primary beams especially at Bands $1$ and $2$ hence, a very useful tool for IM experiments . 
\item If we consider a correct model of the primary beam, then the intrinsic partial leakage of linear polarization ($|Q + iU|_{T} \longrightarrow I$) is given at $\approx 1.0 \%$, 
hence, making it possible to evaluate the angular spectrum of both CO and $21$ cm  at a multipole moment of $l \lesssim 50$  which agrees with when we correct for
the beam errors in Stokes $I$. However, this is contrary if no beam correction is made.
\item Convolution is also a very good technique for IM experiments since we can use this approach to measure total intensity of a signal. 
\item CO IM is more difficult to perform as compared to HI IM, because this type of experiment is observed at high frequencies and when an antenna has a very large gain,
the beamwidth is also very small and the antenna requires very careful control over its position.
\end{itemize}
%%

\noindent Future expansion of this work is as follows:
\begin{itemize}
 \item Use the PCA and Spherical harmonics models of MeerKAT L-band primary beams that are discussed in \citep{asad2018} to perform IM experiment and valid our existing results.
 
 \item Explore other beam distortions schemes such as dish mis-alignment and secondary blockage (supporting frames) to give a better understanding of the primary beam effects on 
 IM.
 
 \item Adopt recent techniques such as deep learning algorithms to also investigate the potential of IM experiments.
\end{itemize}


\noindent  In short, with a good model of the primary beams we can actually estimate the amount of foregrounds that leak from intensity to polarization.

